En la teoría de grafos, el problema del \textit{camino mínimo}\footnote{ Ver Thomas H. Cormen; Charles E. Leiserson; Ronald L. Rivest y Clifford Stein. Introduction to algorithms. 2009. Sección 24: \textit{Single-source shortest paths}.\label{foot_1}} se refiere a una serie de problemas relacionados a encontrar, para un grafo ---o digrafo--- \mbox{$G = (V,\ E)$} con función de peso \mbox{$w : E \to \mathbb{R}$} asociada y ciertos pares de vértices $\mathcal{C}$, un conjunto de caminos $s \rightsquigarrow t$, $(s,\ t) \in \mathcal{C}$, para los cuales la suma total del peso de sus aristas ---su \textit{distancia}--- es mínima de entre todos los caminos posibles con esos extremos. En este informe, nos vamos a concentrar en la variante del problema conocida como \textit{camino mínimo a partir de una única fuente}, donde interesa conocer la distancia de cualquier camino mínimo entre un vértice $s \in V$ y todo el resto de los vértices $w \in V \backslash \{s\}$.

Existen diversos métodos para la resolución de este problema. Entre ellos, los algoritmos \textit{golosos} de \textit{Bellman-Ford} y de \textit{Dijkstra}, que se basan en el concepto de \textit{relajación}\footnote{ Podemos pensar en el proceso de relajación como un método por el cual se mejora, sucesivamente, la cota superior de la distancia que puede tener un camino mínimo. El mismo se basa en la propiedad de desigualdad triangular: si $\delta : E \to \mathbb{R}$ denota la distancia mínima entre cualquier par de vértices en $V$, entonces para cualquier par de vértices $s$ y $t$ y arista $(u,\ t) \in E$ con $u \neq s$,  $\delta(s,\ t) \leq \delta(s,\ u) + w(u,\ t)$.} de aristas para la construcción de una solución.  

El siguiente informe evalúa el problema del \textit{tráfico}, explicado en el próximo apartado, y lo reformula como una aplicación particular del problema de \textit{camino mínimo a partir de una única fuente} que aprovecha la propiedad de subestructura óptima de los caminos mínimos. Además, evalúa la eficiencia de la solución propuesta de manera empírica. %en función de la aplicación de diferentes heurísticas. En particular, \textit{union by rank} y \textit{path compression}\footnote{ Ver nota al pie (\ref{foot_1}). Sección 21: \textit{Data Structures for Disjoint Sets}.\label{foot_3}}.

$\\$
\noindent Palabras clave: \textit{camino mínimo, algoritmo de Dijkstra}.
